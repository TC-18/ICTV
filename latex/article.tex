\documentclass{llncs}
\usepackage{enumerate}% http://ctan.org/pkg/enumerate
\usepackage{multirow}
\usepackage{amsmath,amssymb}
\usepackage{url}
\usepackage{overpic}
\usepackage{enumerate}
\usepackage{graphicx}        % standard LaTeX graphics tool
\usepackage{tikz}        % standard LaTeX graphics tool
%\usetikzlibrary{arrows}
%\usetikzlibrary{quotes,angles}
\usepackage{subfigure}                                 % authors: subfigures
\usepackage[ruled,vlined,linesnumbered]{algorithm2e}   % authors: last version of algorithm display
\usepackage{todonotes}



\newcommand{\ie}{\emph{i.e.} }
\newcommand{\eg}{\emph{e.g.} }
\newcommand{\wnlog}{w.l.o.g. }
\newcommand{\Zr}{\ensuremath{\mathbb{Z}[\rho]}}
\newcommand{\C}{\ensuremath{\mathbb{C}}}
\newcommand{\E}{\ensuremath{\mathcal{E}}}

\title{Interactive Curvature Tensor Visualization on Digital
Surfaces\thanks{This work has been mainly funded by XXXXXXX research grants.}}

\author{H\'el\`ene Perrier\inst{1}\and J\'eré\'emy Levallois\inst{1,2}\and David
Coeurjolly\inst{1}\and Jean-Philippe Farrugia\inst{1}\and Jean-Claude
Iehl\inst{1}\and Jacques-Olivier Lachaud\inst{2} }

%\address[liris]{Universit\'e de Lyon, CNRS, INSA-Lyon, LIRIS, UMR5205, F-69621, France}
%\address[lama]{Universit\'e de Savoie, CNRS, LAMA, UMR 5127, F-73776, France}

 \institute{ Universit\'e de Lyon, CNRS\\
   LIRIS, UMR5205, F-69621, France
%   \email{\{david.coeurjolly,jeremy.levallois\}@liris.cnrs.fr}
   \and
Universit\'e de Savoie, CNRS\\
LAMA, UMR5127, F-73776, France\\
%\email{jacques-olivier.lachaud@univ-savoie.fr}
}

\graphicspath{{./Figs/}}
%fonts bonanza
\usepackage{amsmath,amssymb,amsfonts}
\usepackage{pifont}% http://ctan.org/pkg/pifont
\newcommand{\CheckMark}{\ding{51}}%
\newcommand{\CrossMark}{\ding{55}}%
% Zapf font
\usepackage[mathscr]{euscript}
\DeclareFontFamily{OT1}{pzc}{}
\DeclareFontShape{OT1}{pzc}{m}{it}%
              {<-> s * [1.2] pzcmi7t}{}
\DeclareMathAlphabet{\mathpzc}{OT1}{pzc}{m}{it}
% rescaling cal to be a touch smaller
\DeclareFontFamily{OMS}{fcmsy}{\skewchar\font48 }
\DeclareFontShape{OMS}{fcmsy}{m}{n}{%
         <-5.5> [.96] cmsy5     <5.5-6.5> [.96] cmsy6
      <6.5-7.5> [.96] cmsy7     <7.5-8.5> [.96] cmsy8
      <8.5-9.5> [.96] cmsy9     <9.5->  [.96] cmsy10
      }{}
\DeclareFontShape{OMS}{fcmsy}{b}{n}{%
       <-6> [.96] cmbsy5
      <6-8> [.96] cmbsy7
      <8->  [.96] cmbsy10
      }{}
\DeclareMathAlphabet{\mathcal}{OMS}{fcmsy}{m}{n}
%\usepackage{bbm}

\usepackage{mathtools}% http://ctan.org/pkg/mathtools
\usepackage{calc}% http://ctan.org/pkg/calc

\newcommand*{\mytilde}[2][0pt]{%
  \setbox0=\hbox{$#2$}%
  \tilde{\mathrlap{\phantom{\rule{\wd0}{\ht0+{#1}}}}\smash{#2}}%
}
\newcommand*{\mywidetilde}[2][0pt]{%
  \setbox0=\hbox{$#2$}%
  \widetilde{\mathrlap{\phantom{\rule{\wd0}{\ht0+{#1}}}}\smash{#2}}%
}

\newtheorem{Definition}{Definition}
\newtheorem{Theorem}{Theorem}
\newtheorem{Proposition}{Proposition}
\newtheorem{Corollary}{Corollary}
\newtheorem{Problem}{Problem}
\newtheorem{Lemma}{Lemma}
\newtheorem{Claim}{Claim}

%%Space, Lattices
\newcommand{\Z}{{\mathbb{Z}}}
\newcommand{\N}{{\mathbb{N}}}
\newcommand{\R}{{\mathbb{R}}}
\newcommand{\M}{\mathcal{M}}
\newcommand{\B}{\mathcal{B}_R}

\newcommand{\BT}[1]{\ensuremath{\partial #1}}
\newcommand{\Bd}[1]{\ensuremath{\partial #1}}
\newcommand{\dS}{\BT X}
\newcommand{\Body}[2]{\ensuremath{\lbrack #1 \rbrack_{#2}}}

\newcommand{\Shape}{\ensuremath{X}}
\newcommand{\DigShape}{\ensuremath{Z}}
\newcommand{\Boundary}[1]{\ensuremath{\partial #1}}
\newcommand{\DigBoundary}[1]{\ensuremath{Bd(#1)}}
\newcommand{\Shapes}{\ensuremath{\mathbb{X}}}
\newcommand{\vx}{\ensuremath{\mathbf{x}}}
\newcommand{\vxH}{\ensuremath{\hat{\mathbf{x}}}}
\newcommand{\vp}{\ensuremath{\mathbf{p}}}
\newcommand{\vw}{\ensuremath{\mathbf{w}}}
\newcommand{\vn}{\ensuremath{\mathbf{n}}}
\newcommand{\Dig}{\ensuremath{\mathtt{G}}}
\newcommand{\DigF}[2]{\ensuremath{\Dig_{#2}\left(#1\right)}}
\newcommand{\DSh}{\DigF{\Shape}{h}}

\newcommand{\MCard}{\ensuremath{\mathrm{Card}}}
\newcommand{\Area}{\ensuremath{\mathrm{Area}}}
\newcommand{\Vol}{\ensuremath{\mathrm{Vol}}}
\newcommand{\AreaC}[0]{\ensuremath{\widehat{\Area}}}
\newcommand{\VolC}[0]{\ensuremath{\widehat{\Vol}}}
\newcommand{\Ball}[2]{\ensuremath{B_{#1}\left(#2\right)}}

%% Curvature notations
\newcommand{\Curv}{\ensuremath{\kappa}}
\newcommand{\MeanCurv}{\ensuremath{H}}
\newcommand{\GaussCurv}{\ensuremath{K}}
\newcommand{\PrincCurv}[1]{\ensuremath{\kappa_{#1}}}
\newcommand{\PrincDir}[1]{\ensuremath{\vw_{#1}}}
\newcommand{\NormalDir}{\ensuremath{\vn}}

%% Pottmann curvature estimators
\newcommand{\CurvT}[1]{\ensuremath{\tilde{\Curv}^{#1}}}
\newcommand{\MeanCurvT}[1]{\ensuremath{\tilde{\MeanCurv}^{#1}}}
\newcommand{\GaussCurvT}[1]{\ensuremath{\tilde{\GaussCurv}^{#1}}}
\newcommand{\PrincCurvT}[2]{\ensuremath{\tilde{\PrincCurv{}}_{#1}^{#2}}}
\newcommand{\PrincDirT}[2]{\ensuremath{\tilde{\PrincDir{}}_{#1}^{#2}}}
\newcommand{\NormalDirT}[1]{\ensuremath{\tilde{\NormalDir{}}^{#1}}}

%% II curvature estimators
\newcommand{\CurvH}[1]{\ensuremath{\hat{\Curv}^{#1}}}
\newcommand{\MeanCurvH}[1]{\ensuremath{\hat{\MeanCurv}^{#1}}}
\newcommand{\GaussCurvH}[1]{\ensuremath{\hat{\GaussCurv}^{#1}}}
\newcommand{\PrincCurvH}[2]{\ensuremath{\hat{\PrincCurv{}}_{#1}^{#2}}}
\newcommand{\PrincDirH}[2]{\ensuremath{\hat{\PrincDir{}}_{#1}^{#2}}}
\newcommand{\NormalDirH}[1]{\ensuremath{\hat{\NormalDir{}}^{#1}}}

%%Formulas
\newcommand{\EqDef}{\!\ensuremath{\mathrel{\mathop:}=}\!}
%\newcommand{\EqDef}{\smash{\ensuremath{\stackrel{\text{def}}{=}}}}

%% Misc.
\newcommand{\txtblue}[1]{\textcolor{blue}{ #1}}
\newcommand{\txtgreen}[1]{\textcolor{green}{ #1}}
\newcommand{\txtred}[1]{\textcolor{red}{ #1}}


\DeclareMathOperator*{\argmin}{arg\,min}

\begin{document}
\maketitle


\begin{abstract}\sloppy
  Interactive visualization is a very convenient tool to explore
  complex scientific data or to explore different parameter settings
  for a given processing algorithm. In this article, we present a tool
  to efficiently explore the curvature tensor on the boundary of
  potentially large digital objects. More precisely, we combine a
  fully parallel pipeline on GPU to extract an adaptive triangulated
  iso-surface of the digital object, with a curvature tensor
  estimation at each surface point based on integral
  invariants. Integral invariants being parametrized by a given ball
  radius, our proposal allows us to explore interactively different
  radii and thus select the appropriate scale to which the computation
  is performed and visualized (mean and gaussian curvature, principal
  curvatures, principal directions and normal vector field).


\keywords{Isosurface Visualization, Digital Geometry, Curvature
  Estimation, GPU.}
\end{abstract}

\section{Introduction}
\label{sec:introduction}



\textbf{Contributions}


\section{Preliminaries}
\label{sec:preliminaries}

\subsection{Curvature Tensor Estimation}
\label{sec:curv-tens-estim}

A {\em cell} $(k,x,y,z)$ is a region of the space of size $2^k \times
2^k \times 2^k$, characterized by its integer coordinates $x,y,z$,
with $0 \le x < 2^k$, $0 \le y < 2^k$, $0 \le z < 2^k$. Cells forms an
octree decomposition of a cubic space. We use functions Up, Down and
Next to navigate between cells.


\begin{algorithm}
\KwIn{Integers $p,q,r$ \tcp*{the moment orders, with $0 \le p+q+r \le 2$}}
\KwIn{Integer $k$ \tcp*{$(2^k)^3$ is the size of the digital shape image}}
\KwIn{Mipmap $V$ \tcp*{array of $k+1$ images of sizes $(2^k)^3,  (2^{k-1})^3, \ldots,  1^3$}}
\KwIn{Integers $x_0,y_0,z_0$, Real $r$ \tcp*{Ball radius $r$ and center $(x_0,y_0,z_0)$}}
\KwOut{Real $m$ \tcp*{estimation of the $p,q,r$-moment of $X \cap B_r(x_0,y_0,z_0)$}}
\KwData{Cell $c :=  (k,0,0,0)$ \tcp*{Starts from biggest cell}}
\KwData{Integer $n := 2^k$ \tcp*{Size of each cell}}
\KwData{Real $d,\delta,l$ \tcp*{variables for intermediate computations}}
\Begin{
    m := 0 \;
    \Repeat{$c[0] = k$}{
      \tcp{Distance between cell and ball centers}
      $d := \frac{1}{2}\| 2^k(2c[1]+1,2c[2]+1,2c[3]+1) - (2x_0+1,2y_0+1,2z_0+1) \|_2$\;
      $\delta := \frac{\sqrt{3}}{2}2^{c[0]}$ \tcp*{half-length of cell diagonal}
      \If(\tcp*[f]{Is it a smallest cell ?}){$c[0] = 0$}{
        \If(\tcp*[f]{Is cell inside ball ?}){$d^2 \le r^2$}{
          $m := m + V[c] * (2^k c[1])^p * (2^k c[2])^q * (2^k c[3])^r$
        }
        $c := \textsc{Next}(c)$ \tcp*{Go to next cell}
      } \Else{
        $l := \max(r-\delta,0)$\;
        \If(\tcp*[f]{Is cell completely inside ball ?}){$d^2 < l^2$}{
          $m := m + V[c] * (2^k c[1])^p * (2^k c[2])^q * (2^k c[3])^r$
          $c := \textsc{Next}(c)$ \tcp*{Go to next cell}
        }\ElseIf(\tcp*[f]{Is cell outside ball ?}){$d^2 > (r+\delta)^2$}{
          $c := \textsc{Next}(c)$ \tcp*{Go to next cell}
        }\lElse(\tcp*[f]{Go to a finer cell}){$c := \textsc{Down}(c)$}
      }
    }
    \Return{m}\;
  }
  \caption{Derecursified hierarchical algorithm for computing the
    $p,q,r$-moment of set $X \cap B_r(x_0,y_0,z_0)$, given a mipmap
    $V$ that represents the volume of a shape $X$ in each cell, and
    ball parameters $x_0,y_0,z_0,r$.}
\end{algorithm}

\begin{function}
  \caption{Up( Cell $c$ ) : Cell}
  \Return{Cell( $c[0]+1$, $c[1]/2$, $c[2]/2$, $c[3]/2$ )}
\end{function}
\begin{function}
  \caption{Down( Cell $c$ ) : Cell}
  \Return{Cell( $c[0]-1$, $c[1]*2$, $c[2]*2$, $c[3]*2$ )}
\end{function}
\begin{function}
  \caption{Next( Cell $c$ ) : Cell}
  \lWhile{$\mathrm{Odd}(c[1])$ and $\mathrm{Odd}(c[2])$ and $\mathrm{Odd}(c[3])$}{$c := \mathrm{Up}(c)$}
  \lIf{$\mathrm{Even}(c[1])$}{$c[1] := c[1]+1$}
  \Else{$c[1] := c[1] - 1$\;
    \lIf{$\mathrm{Even}(c[2])$}{$c[2] := c[2]+1$}
    \Else{$c[2] := c[2] - 1$\;
      $c[3] := c[3]+1$}
  }
  \Return{c}
\end{function}


\subsection{Isosurface Extraction on GPU}
\label{sec:isos-extr-gpu}

\section{Interactive Visualization on GPU}
\label{sec:inter-visu-gpu}

\section{Conclusion and Discussion}
\label{sec:discussion}



\bibliographystyle{splncs03}
\bibliography{ictv}
\end{document}
