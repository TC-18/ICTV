%fonts bonanza
\usepackage{amsmath,amssymb,amsfonts}
\usepackage{pifont}% http://ctan.org/pkg/pifont
\newcommand{\CheckMark}{\ding{51}}%
\newcommand{\CrossMark}{\ding{55}}%
% Zapf font
\usepackage[mathscr]{euscript}
\DeclareFontFamily{OT1}{pzc}{}
\DeclareFontShape{OT1}{pzc}{m}{it}%
              {<-> s * [1.2] pzcmi7t}{}
\DeclareMathAlphabet{\mathpzc}{OT1}{pzc}{m}{it}
% rescaling cal to be a touch smaller
\DeclareFontFamily{OMS}{fcmsy}{\skewchar\font48 }
\DeclareFontShape{OMS}{fcmsy}{m}{n}{%
         <-5.5> [.96] cmsy5     <5.5-6.5> [.96] cmsy6
      <6.5-7.5> [.96] cmsy7     <7.5-8.5> [.96] cmsy8
      <8.5-9.5> [.96] cmsy9     <9.5->  [.96] cmsy10
      }{}
\DeclareFontShape{OMS}{fcmsy}{b}{n}{%
       <-6> [.96] cmbsy5
      <6-8> [.96] cmbsy7
      <8->  [.96] cmbsy10
      }{}
\DeclareMathAlphabet{\mathcal}{OMS}{fcmsy}{m}{n}
%\usepackage{bbm}

\usepackage{mathtools}% http://ctan.org/pkg/mathtools
\usepackage{calc}% http://ctan.org/pkg/calc

\newcommand*{\mytilde}[2][0pt]{%
  \setbox0=\hbox{$#2$}%
  \tilde{\mathrlap{\phantom{\rule{\wd0}{\ht0+{#1}}}}\smash{#2}}%
}
\newcommand*{\mywidetilde}[2][0pt]{%
  \setbox0=\hbox{$#2$}%
  \widetilde{\mathrlap{\phantom{\rule{\wd0}{\ht0+{#1}}}}\smash{#2}}%
}

\newtheorem{Definition}{Definition}
\newtheorem{Theorem}{Theorem}
\newtheorem{Proposition}{Proposition}
\newtheorem{Corollary}{Corollary}
\newtheorem{Problem}{Problem}
\newtheorem{Lemma}{Lemma}
\newtheorem{Claim}{Claim}

%%Space, Lattices
\newcommand{\Z}{{\mathbb{Z}}}
\newcommand{\N}{{\mathbb{N}}}
\newcommand{\R}{{\mathbb{R}}}
\newcommand{\M}{\mathcal{M}}
\newcommand{\B}{\mathcal{B}_R}

\newcommand{\BT}[1]{\ensuremath{\partial #1}}
\newcommand{\Bd}[1]{\ensuremath{\partial #1}}
\newcommand{\dS}{\BT X}
\newcommand{\Body}[2]{\ensuremath{\lbrack #1 \rbrack_{#2}}}

\newcommand{\Shape}{\ensuremath{X}}
\newcommand{\DigShape}{\ensuremath{Z}}
\newcommand{\Boundary}[1]{\ensuremath{\partial #1}}
\newcommand{\DigBoundary}[1]{\ensuremath{Bd(#1)}}
\newcommand{\Shapes}{\ensuremath{\mathbb{X}}}
\newcommand{\vx}{\ensuremath{\mathbf{x}}}
\newcommand{\vz}{\ensuremath{\mathbf{z}}}
\newcommand{\vxH}{\ensuremath{\hat{\mathbf{x}}}}
\newcommand{\vp}{\ensuremath{\mathbf{p}}}
\newcommand{\vw}{\ensuremath{\mathbf{w}}}
\newcommand{\vn}{\ensuremath{\mathbf{n}}}
\newcommand{\Dig}{\ensuremath{\mathtt{G}}}
\newcommand{\DigF}[2]{\ensuremath{\Dig_{#2}\left(#1\right)}}
\newcommand{\DSh}{\DigF{\Shape}{h}}

\newcommand{\MCard}{\ensuremath{\mathrm{Card}}}
\newcommand{\Area}{\ensuremath{\mathrm{Area}}}
\newcommand{\Vol}{\ensuremath{\mathrm{Vol}}}
\newcommand{\AreaC}[0]{\ensuremath{\widehat{\Area}}}
\newcommand{\VolC}[0]{\ensuremath{\widehat{\Vol}}}
\newcommand{\Ball}[2]{\ensuremath{B_{#1}\left(#2\right)}}

%% Curvature notations
\newcommand{\Curv}{\ensuremath{\kappa}}
\newcommand{\MeanCurv}{\ensuremath{H}}
\newcommand{\GaussCurv}{\ensuremath{K}}
\newcommand{\PrincCurv}[1]{\ensuremath{\kappa_{#1}}}
\newcommand{\PrincDir}[1]{\ensuremath{\vw_{#1}}}
\newcommand{\NormalDir}{\ensuremath{\vn}}

%% Pottmann curvature estimators
\newcommand{\CurvT}[1]{\ensuremath{\tilde{\Curv}^{#1}}}
\newcommand{\MeanCurvT}[1]{\ensuremath{\tilde{\MeanCurv}^{#1}}}
\newcommand{\GaussCurvT}[1]{\ensuremath{\tilde{\GaussCurv}^{#1}}}
\newcommand{\PrincCurvT}[2]{\ensuremath{\tilde{\PrincCurv{}}_{#1}^{#2}}}
\newcommand{\PrincDirT}[2]{\ensuremath{\tilde{\PrincDir{}}_{#1}^{#2}}}
\newcommand{\NormalDirT}[1]{\ensuremath{\tilde{\NormalDir{}}^{#1}}}

%% II curvature estimators
\newcommand{\CurvH}[1]{\ensuremath{\hat{\Curv}^{#1}}}
\newcommand{\MeanCurvH}[1]{\ensuremath{\hat{\MeanCurv}^{#1}}}
\newcommand{\GaussCurvH}[1]{\ensuremath{\hat{\GaussCurv}^{#1}}}
\newcommand{\PrincCurvH}[2]{\ensuremath{\hat{\PrincCurv{}}_{#1}^{#2}}}
\newcommand{\PrincDirH}[2]{\ensuremath{\hat{\PrincDir{}}_{#1}^{#2}}}
\newcommand{\NormalDirH}[1]{\ensuremath{\hat{\NormalDir{}}^{#1}}}

%%Formulas
\newcommand{\EqDef}{\!\ensuremath{\mathrel{\mathop:}=}\!}
%\newcommand{\EqDef}{\smash{\ensuremath{\stackrel{\text{def}}{=}}}}

%% Misc.
\newcommand{\txtblue}[1]{\textcolor{blue}{ #1}}
\newcommand{\txtgreen}[1]{\textcolor{green}{ #1}}
\newcommand{\txtred}[1]{\textcolor{red}{ #1}}


\DeclareMathOperator*{\argmin}{arg\,min}
