\section{Conclusion}

\paragraph{}
We presented a new tool for real-time visualisation of volumetric data.
Our method runs entirely and efficiently on the GPU.
We also triangulate the surface on the fly and therefore are able to visualize dynamic scenes such as moving metaballs.
Furthermore, with a LoD criterion that is based on the cells projected size, we prevent geometrical aliasing artefacts during rasterization.

\paragraph{}
One of the main characteristics of our solution is that it is based on MC.
Thus, it does a geometrical sampling of the object.
This implies that we can only render objects that can be sampled, \textit{i.e.} it prevents the rendering of frequencies that exceed by far the Nyquist limit.
To render such objects, ray casting based method are usually used.
Furthermore, the constant re-meshing of the object may also create popping artefacts, that could be fixed with vertex interpolation.

\paragraph{}
The choice of the LoD criterion could also improve this solution.
The actual one is based solely on the projected size of the cell. 
By combining it with the curvature of the object, we could limit the geometrical sampling issues.
However, this requires to be able to measure the curvature without pre process and with a data parallel algorithm.

